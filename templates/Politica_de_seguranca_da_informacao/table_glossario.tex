\begin{longtable}{|>{\centering\arraybackslash}m{3cm}|>{\raggedright\arraybackslash}m{12cm}|}
\hline
\textbf{TERMO} & \textbf{DESCRIÇÃO} \\
\hline
\endfirsthead

\hline
\textbf{TERMO} & \textbf{DESCRIÇÃO} \\
\hline
\endhead

\hline
\endfoot

Backup & Cópia de Segurança. Armazenagem das informações em mídia destinada à contingência de dados. \\
\hline
Backup Site & CPD Secundário. Centro de Processamento de Dados Secundário - Ambiente “Cópia de Segurança” secundário conectado na forma de “espelhamento remoto de dados” com o CPD principal e tem por objetivo entrar em operação imediatamente em casos de paralisação do CPD Primário. É também conhecido como “Hot Backup”. \\
\hline
Firewall & 1- Sistema destinado à proteção lógica de uma rede. 2- Software e/ou hardware que fornece segurança ao isolar a rede de computador de uma empresa do resto da Internet. Um firewall também impede que pessoas de fora conectem a rede; quem estiver interessado em manter o sigilo da informação deve ter um.  
*Fonte: Glossário de TI em \url{http://www.eb.ufmg.br/bax/TecnoInfo/glossario/TI.htm} \\
\hline
Password & Senha. Chave de acesso para um sistema. \\
\hline
No-Break & Apenas no Brasil esse produto é conhecido como No-Break, em todo o mundo ele é conhecido como UPS (Uninterruptible Power Supply). E é exatamente isso que ele é, uma fonte de alimentação ininterrupta de energia elétrica. Seu principal objetivo é proteger sua carga (PCs, Servidores, Periféricos ou outros equipamentos eletrônicos) contra defeitos ou falhas no fornecimento da energia elétrica. \\
\hline
Worms & Um worm, tal como um vírus, foi desenvolvido para copiar a si próprio de computador em computador, mas um worm faz isso automaticamente a apoderar-se de funcionalidades no computador capazes de transportar arquivos ou informações. Um worm que esteja no seu sistema consegue propagar-se sem intervenção do usuário. Um dos grandes perigos dos worms é a sua capacidade de se replicar em grandes volumes, podendo consumir memória, fazendo com que um computador pare de funcionar corretamente. \\
\hline
SPAM & SPAM é o termo usado para se referir aos e-mails não solicitados, que geralmente são enviados para um grande número de pessoas. Quando o conteúdo é exclusivamente comercial, este tipo de mensagem também é referenciado como UCE (do inglês Unsolicited Commercial Email).  
Fonte: \url{http://www.nbso.nic.br/docs/cartilha/cartilha-06-spam.html} \\
\hline
Phishing Scam & Phishing é uma técnica utilizada para obter informação pessoal com propósitos de furto ou fraude através de uso fraudulento de mensagens de emails que aparentam ser de fontes legítimas. Estas mensagens aparentemente autênticas são criadas para enganar os destinatários para divulgarem informações pessoais, tais como: senhas de banco, nºs de cartão de crédito e etc.  
Fonte: \url{http://www.computerworld.com/securitytopics/security/story/0,10801,89096,00.html} e  
\url{http://www.microsoft.com/athome/security/email/phishing.mspx} \\
\hline
Pen drive & Pen drive é um dispositivo com memória flash e conector USB que funciona como unidade de armazenamento removível. Basta plugá-la na porta USB do computador para que seja reconhecida como uma nova unidade de armazenamento pronta para ser utilizada.  
A memória pen drive também é conhecida por outros nomes: memory key, chaveiro USB, flash drive, flash memory, mini HD, entre outros. No Brasil é popularmente conhecida pelo nome de pen drive.  
A grande vantagem desse dispositivo é ser compacto (tamanho aproximado de um chaveiro) com a possibilidade de ter uma grande capacidade de armazenamento, você pode transportá-la para qualquer lugar e plugá-la em qualquer computador com uma porta USB. Podemos dizer que a pen drive tem a função de um mini HD removível.  
Fonte: \url{http://www.pendrivenet.com.br/} \\
\hline
Dispositivo móvel & A definição de dispositivo móvel é qualquer equipamento ou periférico eletrônico, operando com bateria ou não, que possa ser transportado com facilidade e tenha a capacidade de processar, armazenar ou transmitir dados e esteja acessível em qualquer lugar.  
Exemplos de dispositivos móveis: laptops, notebooks, personal digital assistants (PDAs), Blackberry, dispositivos USB (Universal Serial Bus), dispositivos portáteis de rede sem fio (handheld wireless devices), dispositivos USB (Universal Serial Bus), Compact Discs (CDs), Digital Versatile Discs (DVDs), flash drives, MP3/MP4 players, câmeras digitais, telefones com câmera. \\



\end{longtable}