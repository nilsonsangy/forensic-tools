\documentclass[a4paper,12pt]{article}
\usepackage[utf8]{inputenc}
\usepackage[T1]{fontenc}
\usepackage{lmodern}
\usepackage{geometry}
\usepackage{array}
\usepackage{hyperref}
\usepackage{amsmath}
\usepackage{longtable}
\usepackage{array}
\usepackage{hyperref}
\usepackage{breakurl}
\usepackage{ragged2e}
\usepackage{nameref}
\usepackage{cleveref}

\geometry{top=2cm, bottom=2cm, left=2.5cm, right=2.5cm}

\title{\textbf{Política de Segurança da Informação}}
\author{Nilson Sangy Junior} % Kudos ao Bruno que produziu a versão em docx.
\date{\today}

% Definindo um contador para o quarto nível de subseção
\newcounter{subsubsubsection}[subsubsection]  % contador depende do subsubsection
\renewcommand{\thesubsubsubsection}{\thesubsubsection.\arabic{subsubsubsection}}  % formatação da numeração
\newcommand{\subsubsubsection}[1]{%
  \refstepcounter{subsubsubsection}%
  \paragraph{\thesubsubsubsection\ #1}  % formatação do título
}

\renewcommand{\arraystretch}{1.5} % Melhora o espaçamento entre as linhas

\begin{document}

\maketitle

\tableofcontents
\newpage

\section{Objetivo}

Estabelecer diretrizes e objetivos de Segurança da Informação apropriados ao contexto de negócios e seus riscos inerentes, e em acordo com a Missão, Visão e Valores da Financeira XPTO.

\section{Divulgação}

Este documento pode ser encontrado nos locais seguintes:

\begin{itemize}
    \item Em formato digital na Intranet da Empresa XPTO.
    \item Impresso na Diteroria de Tecnologia da Informação.
\end{itemize}

\section{Vigência}

Este normativo deverá ser revisto periodicamente, quando do vencimento de sua vigência, ou quando necessário.

\section{Processo de Referência}

\begin{itemize}
    \item Administração de Segurança da Informação.
\end{itemize}

\section{Documentos Complementares Aplicáveis}

\begin{itemize}
    \item Norma ISO/IEC 27001:2013;
    \item Norma ISO/IEC 27001:2005;
    \item Norma ISO/IEC 27002:2013;
    \item Norma ISO/IEC 27002:2005;
    \item IN2020 Gestão de Mudanças;
    \item IN9876 Confidencialidade e Privacidade no Uso de Informações;
    \item IN3443 Gestão de Eventos e Problemas;
    \item IN3333 Segurança e Proteção Física;
    \item Política de Continuidade de Negócio;
    \item IN3245 Utilização do E-mail Corporativo na FINANCEIRA XPTO;
    \item IN5443 Utilização de Acesso à Internet;
    \item IN1229 Manual do Sistema de Gestão da Segurança da Informação;
    \item IN2013 Segurança de Acesso em Sistemas Críticos.
\end{itemize}

\section{Objetivos da Segurança da Informação e Conceitos}

\subsection{A Segurança da Informação visa atingir os seguintes objetivos}

\begin{itemize}
    \item Desenvolver o entendimento organizacional para gerenciar o risco de segurança sistemas, recursos e dados.
    \item Desenvolver e implementar as salvaguardas adequadas para garantir a entrega de serviços críticos de infraestrutura.
    \item Desenvolver e implementar as atividades adequadas para identificar a ocorrência de um evento de segurança cibernética.
    \item Desenvolver e implementar as atividades apropriadas para agir em relação a um evento detectado de segurança cibernética.
    \item Desenvolver e implementar as atividades apropriadas para manter planos para resiliência e para restaurar quaisquer capacidades ou serviços que foram prejudicados devido a um evento de segurança cibernética.
\end{itemize}

Estes objetivos sempre devem ser avaliados em termos dos custos incorridos para garantir o seu cumprimento e do impacto na eficiência das operações geralmente associado às medidas adicionais de segurança.

As normas de segurança a seguir estabelecidas seguem o padrão definido pelas normas técnicas ABNT- NBR ISO/IEC 27001:2013.

Mais detalhes sobre os objetivos de Segurança da Informação podem ser encontrados no documento “ING002”.

\subsection{Política de Segurança da Informação da Financeira XPTO}

A Política de Segurança da Informação da Financeira XPTO é um conjunto de diretrizes, normas e procedimentos que tem a finalidade de assegurar a disponibilidade, confidencialidade e integridade de seus ativos da informação.

A Política de Segurança da Informação da Financeira XPTO deverá ser aprovada pela Liderança (representada por um ou mais membros do Colegiado da Financeira XPTO), e divulgada para toda empresa a cada nova alteração e/ou revisão. 

Devem estar em conformidade com esta política os funcionários, clientes, prestadores de serviços e fornecedores\footnote{As partes externas devem cumprir os requisitos desta Política mesmo não tendo acesso a ela, sendo de responsabilidade da Financeira XPTO exigir os cumprimentos dessas diretrizes através de cláusulas contratuais, acordos e mecanismos técnicos.}.

A política de segurança da informação da Financeira XPTO é exercitada com atenção nas: Pessoas, Normas, Processos, Tecnologia, Leis e Regulamentos Aplicáveis, Compliance, Auditoria e Sistema de Gestão de Riscos.

\subsubsection{A Política de Segurança da Informação determina:}

\begin{itemize}
    \item Identificar e classificar os riscos dos ativos da informação da organização através do mapeamento das vulnerabilidades, ameaças, impacto e probabilidade de ocorrência, bem como a adoção dos controles que minimizem estes riscos de Segurança da Informação.
    \item Desenvolver um programa de capacitação para os funcionários e contratados, de forma a conscientizar sobre as responsabilidades de cada um em relação à segurança da informação.
    \item Para os novos funcionários, prestadores de serviços e estagiários, a capacitação em Segurança da Informação deve ser aplicada preferencialmente antes do início de suas atividades, sendo de responsabilidade de seu gestor a supervisão durante este período.
    \item Estabelecer e divulgar a responsabilidade de todos pelo zelo e cumprimento das demais políticas e procedimentos de segurança da informação.
    \item Adotar mecanismos de controle e gerenciamento de Segurança da Informação nos serviços de \textit{Outsourcing}, com base nos acordos definidos, designando responsabilidades, penalidades e níveis de acordo de serviços.
    \item Adotar mecanismos automatizados, sempre que possível, para gerenciamento, prevenção e detecção de eventos de segurança.
    \item Efetuar auditorias e inspeções de segurança periódicas, bem como, quando necessário, acompanhar os planos de ação de correção.
    \item Definir mecanismos de proteção física e ambiental a serem implementados pela organização, para prevenir danos e acessos não autorizados a recursos e informações.
    \item Adotar mecanismos para proteção da segurança lógica, a fim de prevenir danos e acessos não autorizados à informação.
    \item Adotar processos de autenticação e controle de acesso seguro para os sistemas de informações.
    \item Apoiar mudanças organizacionais a fim de garantir os aspectos de disponibilidade, integridade e confidencialidade da informação.
    \item Adotar ferramentas de proteção contra softwares maliciosos, vírus, \textit{spam}, \textit{phishing scam} e outros dispositivos que possam ameaçar os sistemas de informação da organização.
    \item Participar das definições e testes do Plano de Continuidade de Negócios, a fim de garantir a disponibilidade dos sistemas de informações.
    \item Estabelecer normas e procedimentos onde todos os funcionários e terceiros devem obrigatoriamente registrar eventos e incidentes de segurança, bem como as respectivas penalidades pelo não cumprimento deste objetivo.
    \item Conscientizar todos os funcionários e terceiros sobre a importância da Segurança da Informação, e a necessidade de seguir as Políticas, Normas, Procedimentos e Instruções referentes ao Sistema de Gestão de Segurança da Informação (SGSI).
    \item Realizar procedimentos periódicos de testes de segurança, como testes de invasão e varredura de vulnerabilidades, a fim de identificar fragilidades nos sistemas e na rede, além de detectar oportunidades de melhoria nos controles de segurança.
\end{itemize}


\subsubsection{Análise Crítica pela Direção}

A alta direção da Financeira XPTO, composta pelo Superintendente Geral e Colegiado, deve analisar criticamente o SGSI. Essa análise deve ser feita por meio de uma reunião que deve ocorrer ao menos uma vez ao ano, com presença obrigatória de ao menos um integrante do Colegiado, de modo que assegure a contínua adequação, pertinência e eficácia do Sistema de Gestão.

A ata de registro desse fórum é utilizada como evidência da execução da “Reunião de Análise Crítica do SGSI” para fins de certificação na norma ISO/IEC 27001:2013.

\subsubsection{Atribuições da Análise Crítica pela Direção}

\begin{itemize}
    \item Situação das ações de análises críticas anteriores;
    \item Mudanças nas questões internas e externas relevantes ao SGSI;
    \item Realimentação sobre o desempenho de Segurança da Informação (pontos de auditorias internas/externas, riscos de segurança, conformidade de controles internos, métricas do SGSI e cumprimento dos objetivos de Segurança da Informação);
    \item Realimentação das partes interessadas;
    \item Oportunidades de melhoria contínua.
\end{itemize}

\subsection{Conceitos Utilizados na Política de Segurança}

\subsubsection{Segurança Física e Ambiental}

Prevenir o acesso físico não autorizado, danos, e interferências nos recursos de processamento das informações e nas informações da organização.

\subsubsection{Segurança Lógica}

Segurança lógica diz respeito ao conjunto de medidas destinadas a assegurar a integridade, confidencialidade e disponibilidade da informação.

\subsubsection{Segurança em Recursos Humanos}

É o aspecto da segurança que define como deve ser a relação dos funcionários e terceiros e partes externas conforme item 8.2.a IN2323 Manual do SGSI com as determinações da Política de Segurança da Financeira XPTO estabelecidas neste documento.

Essa relação se refere às atitudes que a Financeira XPTO espera que os funcionários, terceiros e partes externas adotem frente à segurança da informação, seguindo as determinações e se responsabilizando por elas. 

\subsubsection{Plano de Contingência}

Consiste em procedimentos de recuperação preestabelecidos, com a finalidade de minimizar o impacto sobre as atividades da organização no caso de ocorrência de um dano ou desastre que os procedimentos de segurança não conseguiram evitar.

\section{Disposições Gerais}

\subsection{Planejamento e Implantação da Cultura de Segurança da Informação}

Cabe à Liderança da Financeira XPTO:

\begin{itemize}
    \item Promover e manter sólida cultura em segurança da informação, apoiando as ações de conscientização; incentivar o contínuo aprimoramento dos processos para mitigação dos riscos de vazamento de informações e aprovar os recursos necessários junto ao Conselho de Administração, para que a gestão da segurança da informação tenha eficácia.
    \item Explicitar formalmente para todos os funcionários, terceiros e partes interessadas qual o nível de comprometimento que exige dos empregados e as atitudes esperadas frente às questões de segurança da informação, conforme os passos a seguir.
\end{itemize}

\subsubsection{Planejamento}

\begin{itemize}
    \item Definição da Política de Segurança da informação;
    \item Definir os requisitos básicos de segurança comuns a todos os funcionários atendendo as definições estabelecidas nas normas que tratam de:
    \begin{itemize}
        \item Confidencialidade, disponibilidade e privacidade no uso de informações;
        \item Segurança lógica (a rede, sistemas e internet);
        \item Segurança e proteção física;
        \item Acesso físico às dependências.
    \end{itemize}
    \item Esclarecer sobre as atitudes esperadas e responsáveis pelas atividades;
    \item Deixar claro como lidar com procedimentos não especificados e requisitos e situações não previstas;
    \item Definir as posturas a adotar no caso de quebra de segurança (grau de tolerância x penalidades).
    \item Definição do Termo de Responsabilidade
    \item Apoiar-se na Assessoria Jurídica para que o texto cubra todos os aspectos que possam deixar a Financeira XPTO coberta no que diz respeito às questões legais e fiscais, principalmente as relacionadas com as legislações trabalhista e cíveis.
    \item Considerar os aspectos relacionados a quaisquer recursos em poder de empregados tais como: computadores portáteis, Smartphones, informações, credenciais de acesso, senhas, chaves e crachás.
    \item Explicitar o conceito de que “o que não estiver previsto é proibido”.
    \item Explicitar que o empregado vai estar sujeito a penalidades se não respeitar as regras de segurança.
\end{itemize}

\subsubsection{Implantação}

\begin{itemize}
    \item Comunicação Formal da Liderança da Financeira XPTO e Capacitações.
    \item Cabe à Liderança da Financeira XPTO utilizar-se de técnicas que melhor lhe convierem para a implantação de uma cultura de segurança. O foco é buscar a melhor forma de fixar os conceitos, obtendo atitudes esperadas considerando a própria cultura da empresa.
    \item Após a capacitação, colher concordância, de funcionários, terceiros e partes interessadas, quanto às penalidades e controles a que o empregado estiver sujeito (filmagem, monitoramento de terminal, utilização de telefone celular, regras para utilização de telefone e meios de comunicação disponibilizados para realização das atividades).
    \item Monitoração das atividades e promoção dos ajustes necessários.
    \item A monitoração das atividades deverá ser realizada por todas as áreas da Financeira XPTO.
    \item Canal de comunicação para informar desvios fica na Intranet da Financeira XPTO na opção “Evento de Segurança”.
    \item Devem ser informados quaisquer desvios que impactem o bom andamento dos processos da FINANCEIRA XPTO.
\end{itemize}

\subsection{Estruturação da Gestão da Segurança da Informação}

É de competência da Liderança da Financeira XPTO providenciar a instalação e a manutenção de uma estrutura de pessoal para a gestão das atividades voltadas para a segurança da informação.

A função do gestor de segurança da informação é manter e efetuar análise crítica das ocorrências de segurança de acordo com as normas e procedimentos estabelecidos. As avaliações abordarão:

\begin{itemize}
    \item A efetividade da política, decorrentes de volumes e impactos dos eventos e incidentes de segurança registrados;
    \item Efeitos das mudanças de ambientes interno ou externo e tecnologias, processos e produtos.
\end{itemize}

\subsubsection{Segurança Física}

A Segurança Física relaciona-se diretamente com os aspectos associados ao acesso físico a recursos de informações, cabendo às áreas responsáveis a definição e estabelecimento das regras operacionais.

\subsubsubsection{Responsabilidades da Gerência Adjunta Administrativa}

\begin{itemize}
    \item Implantar procedimentos formais de controle de acesso às instalações (INSG006 Acesso Físico às Dependências da Financeira XPTO);
    \item Implantar procedimentos formais de registro e controle de uso de equipamentos (IN0192 Segurança e Proteção Física);
    \item Implantar controles formais de condições ambientais da instalação (IN9743 Segurança e Proteção Física);
    \item Implantar desde que economicamente viáveis dispositivos que minimizem efeitos de flutuações ou interrupções no fornecimento de energia elétrica, (no-break, geradores) (IN1221 Segurança e Proteção Física);
    \item Implantar procedimentos formais de avaliação e manutenção dos equipamentos auxiliares (alimentação energia elétrica, ar condicionado e equipamentos de segurança) (IN3323 Segurança e Proteção Física).
\end{itemize}

\subsubsubsection{Responsabilidades dos usuários}

\begin{itemize}
    \item Cumprir com o procedimento formal de controle de acesso de pessoal, ao ambiente físico da respectiva área, quando aplicável, tais como: quadro de distribuição de telecomunicação, unidades de controle de teleprocessamento, periféricos computacionais da área e as estações de trabalho.
\end{itemize}

\subsubsection{Segurança Lógica}

A política de Segurança Lógica (IN-SG008-2002 – Segurança Lógica) refere-se ao acesso a rede, sistemas e aplicativos que as pessoas utilizam para cumprimento de suas atividades de trabalho.

\subsubsubsection{Responsabilidades da Superintendência Tecnologia da Informação}

\begin{itemize}
    \item Implantar procedimentos formais para a duplicação periódica de arquivos de dados, programas e documentação (Backup) e zelar pela sua guarda em local adequado (IN8982 Segurança Lógica e IN3019 Segurança e Proteção Física).
    \item Implantar procedimentos formais de controles de atualizações da biblioteca de programa em produção (IN2002 Segurança Lógica e IN-SG019-2002 Segurança e Proteção Física).
    \item Implantar na medida das solicitações, os recursos necessários para criptografia de dados transmitidos por meios eletrônicos (IN2002 Segurança Lógica).
    \item Implantar um plano formal de manutenção preventiva de equipamentos (IN0023 Manutenção de Hardware).
\end{itemize}

\subsubsubsection{Responsabilidades dos usuários}

\begin{itemize}
    \item Cumprir as determinações das Políticas, Normas e Procedimentos e em especial essa Política de Segurança.
    \item Avaliar sistemas colocados a serem colocados em produção no que diz respeito ao atendimento das especificações estabelecidas nas diferentes etapas do ciclo de desenvolvimento.
\end{itemize}

\subsubsubsection{Responsabilidades dos Desenvolvedores de Sistemas}

\begin{itemize}
    \item Implantar sistema de registro formal do acervo de programas e arquivos.
    \item Implantar os requisitos de segurança de um sistema especialmente no que diz respeito a critérios de consistência, auditabilidade e rastreabilidade, controles adicionais de acesso via “password” ou outros dispositivos, inclusão de “check-points” para aplicações “batch” de processamento demorado e procedimentos de distribuição e destruição de relatórios.
    \item Minimizar a dependência da intervenção humana de operadores/controladores para a correta operação dos sistemas desenvolvidos.
    \item Considerar que os sistemas a serem colocados em produção correspondam às especificações aprovadas nas diferentes etapas do ciclo de desenvolvimento.
    \item Corrigir prontamente vulnerabilidades e falhas de segurança identificadas por processos de auditoria e análise de qualidade.
\end{itemize}

\subsubsection{Segurança em Recursos Humanos}

A Financeira XPTO manterá uma Cultura de Segurança, informando de maneira estruturada o que espera dos empregados, o que é permitido, tratamento de casos não previstos e as penalidades às quais os funcionários, terceiros e partes interessadas estarão sujeito quando regras forem quebradas. Define claramente qual é a postura da empresa e o nível de tolerância a eventuais desvios de conduta.

Para que a Financeira XPTO possa cobrar essa postura de seus funcionários, terceiros e partes interessadas faz-se necessário o cumprimento dos aspectos:

\subsubsubsection{Responsabilidades da Liderança da FINANCEIRA XPTO}

\begin{itemize}
    \item Assegurar que a Política de Segurança da Informação e os objetivos de segurança da informação estão estabelecidos e são compatíveis com a direção estratégica desta organização (Itens 1 e 6.1 desta norma).
    \item Compromete-se com a melhoria contínua do SGSI. – Sistema de Gestão de Segurança da Informação e dos requisitos desta política.
    \item Firmar acordos de confidencialidade e termos de responsabilidade com empresas contratadas e recursos terceiros, a fim de que sejam cumpridas a Política de Segurança da Informação da Financeira XPTO.
\end{itemize}

\subsubsubsection{Responsabilidades dos Funcionários}

\begin{itemize}
    \item Participar sempre que convocado, dos eventos e capacitações relacionados à Segurança da Informação promovidos pela Financeira XPTO.
    \item Praticar as determinações da Liderança da Financeira XPTO, relacionadas às Políticas, Normas, Procedimentos e manuais. Cumprir os requisitos de segurança da informação.
\end{itemize}

\subsubsection{Plano de Continuidade do Negócio}

O plano de continuidade do negócio engloba Plano de Contingência e Plano de Recuperação. Estes planos consistem em estabelecer os procedimentos de preservação e recuperação de ambientes e sistemas de processamento de dados, minimizando o impacto sobre as atividades da FINANCEIRA XPTO.

Os procedimentos e responsabilidades sob o Plano de Continuidade do Negócio estão descritos no IN0041 Manual do SGCN.

\subsection{Normas Gerais}

\subsubsection{Gerenciamento de Equipamentos do Ambiente de Informática}

\subsubsubsection{Ambiente de rede}

A rede de computadores consiste no meio de tráfego, armazenamento e execução de aplicações e sistemas dentro dos negócios da Financeira XPTO.

Este meio é protegido por níveis de segurança e administrado pela Superintendência Tecnologia da Informação. 

O ambiente de rede é composto por servidores de dados, meios físicos de tráfego de dados, estações de trabalho e ferramentas de gestão dessa infraestrutura.

Toda e qualquer alteração na configuração do hardware e software do ambiente de rede (inclui computadores e seus acessórios) deve ser precedida de avaliação do processo de Gestão de Mudanças, Gestão de Riscos e Segurança de Informações. 

Os cabos e conectores da rede deverão ser instalados e mantidos por técnicos qualificados para garantir sua integridade e funcionalidade, devendo ser previamente testados (quando de sua instalação) a fim de assegurar a conexão entre os equipamentos.

Qualquer ponto de rede não usado deverá estar identificado, desabilitado e formalmente documentado.

\subsubsubsection{Documentação}

A documentação de todos os equipamentos envolvidos com o ambiente de tecnologia de informática deverá ser mantida atualizada e disponível para as pessoas responsáveis e autorizadas a manterem estes equipamentos funcionando adequadamente.

\begin{itemize}
    \item Inventário - um inventário formal de todos os equipamentos e licenças de uso de produtos deverá ser controlado e mantido atualizado.
    \item Falhas em equipamentos - todas as falhas ocorridas com os equipamentos deverão ser prontamente comunicadas e registradas num sistema de registro de falhas. As causas raiz desses incidentes devem ser analisadas e tratadas de forma a evitar a recorrência de falhas.
\end{itemize}

\subsubsection{Comunicação de Dados}

\subsubsubsection{Canais privativos para comunicação de dados}

O negócio da Financeira XPTO, pelo fato dos locais de processamento de dados e de operação dos sistemas estarem distantes entre si, depende fundamentalmente de canais de comunicação. Para assegurar a continuidade do seu negócio, baseado na operação dos sistemas de negócio, a Financeira XPTO deverá manter canais redundantes (backup) para comunicação de dados.

A capacidade e velocidade devem ser compatíveis com as necessidades da Financeira XPTO de forma a garantir uma boa performance da operação. Os canais de comunicação de dados principal e redundante devem obrigatoriamente ser contratados de fornecedores diferentes e que em nenhum momento se sobreponham.

\subsubsection{Configuração das Estações de Trabalho e Inventário de Software}

As estações de trabalho de usuários do ambiente de informática deverão obedecer ao padrão, descrito na norma IN0032 Estação de Trabalho Padrão, estabelecida pela Superintendência Tecnologia da Informação responsável pela configuração dos equipamentos, softwares a serem disponibilizados e pelo cadastramento e permissões de acesso de usuários à rede:

\begin{itemize}
    \item As estações de trabalho padrão devem possuir sistema centralizado para controle de acesso das interfaces de entrada/saída (I/O), por exemplo: disquetes, CD, DVD, PEN-DRIVE etc.;
    \item Os softwares básicos, aplicativos e/ou software de qualquer natureza, serão instalados mediante anuência da Superintendência de Tecnologia da Informação que procederá com o inventário contínuo e o controle de licença dos produtos;
    \item As estações de trabalho poderão ser auditadas de forma aleatória e sem prévio aviso, para identificar a prática de pirataria de direitos autorais de software.
    \item Os trabalhos pessoais como planilhas e textos, considerados sensíveis à perda, deverão ser armazenados em área previamente definida, no ambiente da rede de computadores para que sejam contempladas nos processos de backup.
\end{itemize}

É expressamente proibida a instalação de software, por download diretamente da Internet, exceto no caso de fornecedores contratados que disponibilizam seus softwares para download. É necessária uma autorização expressa das áreas de Infraestrutura e Suporte de TI e Segurança da Informação.

Toda a falha ocorrida na utilização de softwares, quer sejam básicos ou aplicativos, deverá ser prontamente comunicada e registrada num sistema de registro de falhas.

\subsubsection{Confidencialidade e Privacidade no Uso de Informações}

Os recursos de processamento da informação são fornecidos para propósitos do negócio. O uso impróprio ou não autorizado constitui em transgressão às regras estabelecidas, cabendo sanções disciplinares cabíveis.

\subsubsubsection{Uso e descarte de relatórios impressos}

Informações classificadas como confidenciais nunca devem ser impressas numa impressora de rede (impressora compartilhada para uso comum entre os usuários da rede) sem que o emitente ou uma pessoa autorizada resguarde sua confidencialidade durante e após a impressão.

Qualquer relatório impresso, tendo em seu conteúdo, informações confidenciais, deve ser inutilizado imediatamente após seu uso, evitando que terceiros possam acessá-lo. Quando for necessário arquivá-lo ou mesmo guardá-lo para continuidade do trabalho em dia subsequente, nunca o relatório deverá ser deixado sobre a mesa de trabalho ou em gaveta que não garanta sua segurança, conforme trata a norma “IN0192 Segurança e Proteção Física”.

\subsubsubsection{Uso e descarte de mídia de armazenamento removível (CDs, DVDs e PENDRIVE)}

Somente pessoal autorizado a instalar ou modificar software poderá utilizar mídia removível para transferir dados de/para a rede da Financeira XPTO. Preferencialmente esta atividade deverá estar restrita à área de suporte técnico e/ou administradores da rede. 
Informações confidenciais que não serão mais necessárias, armazenadas em meios magnéticos, deverão ser apagadas antes de se disponibilizar a mídia para outra finalidade, conforme trata a norma “IN0128 Descarte de Mídias na Financeira XPTO”.

\subsubsubsection{Uso de dispositivos móveis e portáteis (notebook, telefones, smartphones, PDAs, Tablets, etc.)}

O uso de dispositivos móveis deve sempre ser autorizado conforme recomendações da IN0011 Uso de Dispositivos Móveis na Financeira XPTO.

Os usuários devem aceitar os termos e condições de uso e especialmente responsabilizar-se pela segurança das informações contidas em suas unidades de armazenamento de dados.

\subsection{Proteção das Informações Contra Riscos de Incêndio e Outros}

Todo ativo de informação deve estar protegido contra risco de incêndio durante todo o tempo. O nível da proteção deverá refletir o risco e valor da informação contida no ativo.

\subsection{Acesso à Internet}

O acesso à Internet, conforme trata a norma ``IN2009 Utilização de Acesso à Internet'', é configurado no perfil de rede do usuário requisitante, sendo este acesso pessoal e intransferível, onde o usuário é responsável por este recurso e pelos atos cometidos por ações de empréstimos.

A solicitação do acesso deve ser feita formalmente pelo usuário com a aprovação de seu gerente imediato. 

Este recurso é disponibilizado pela Financeira XPTO, tendo em vista os benefícios oferecidos como apoio às atividades de trabalho. Entretanto, cabe à Gerência Infraestrutura e Suporte TI executar o acompanhamento da sua correta utilização, pois alguns fatores configuram esta necessidade. São eles:

\begin{itemize}
    \item O acesso à internet pode resultar em uma porta de entrada na rede, expondo-a a ataques de vírus, worms, spams, phishing scams e entidades com fins de manipulação ilícita das propriedades intelectuais da Financeira XPTO ou dano à informação;
    \item O acesso ao conteúdo de sites na Internet deve ser controlado através de softwares específicos de análise de conteúdo, permitindo basicamente os sites que tenham relacionamento às atividades de negócio da Financeira XPTO;
    \item A obtenção de arquivos da Internet por meio de download deve ser restrita aos grupos de usuários autorizados e controlados por softwares específicos de análise de conteúdo, assim como verificados e protegidos contra vírus;
    \item A Financeira XPTO se reserva o direito de monitorar o uso da Internet nas estações que compõem a sua rede, através de ferramentas de análise e proteção.
\end{itemize}

\subsection{Criptografia de Dados}

A criptografia é um método utilizado para modificar um texto original de uma mensagem a ser transmitida, gerando um texto inteligível na origem, através de um processo de codificação definido por este método de criptografia.

Em se tratando do negócio da Financeira XPTO, onde a informação operacional é de extrema sensibilidade, a manutenção de processo de criptografia de dados através de software apropriado, é imprescindível para os seus sistemas. Toda transmissão de dados tem que adotar criptografia para dos dados sensíveis, conforme normas “IN8202 Segurança Lógica” e “IN2013 Segurança de Acesso em Sistemas Críticos”.

Sempre que possíveis todas as transmissões de dados sensíveis ou confidenciais devem ser criptografadas e autenticadas com do uso de certificados digitais.

\subsubsection{Chave de Acesso e Senha}

Os usuários de ativos de informações somente poderão utilizá-los por meio de chaves de acesso, autenticadas por senhas secretas de seu exclusivo conhecimento, conforme estabelece a norma IN3382 Segurança Lógica.

\subsubsection{Controle de Acesso Lógico}

\subsubsubsection{Controle de acesso à rede de computadores}  

A definição do perfil de usuários para o acesso à Rede Corporativa (IN0082 Segurança Lógica) caberá à área de Segurança da Informação, bem como estabelecer os perfis de acesso básico por cargo ou função, revisado e aprovado pelos gestores e gestor proprietário da informação.

Os acessos realizados deverão ser registrados em arquivos de Log, protegidos adequadamente e monitorados a fim de identificar potencial mau uso de sistemas ou informações.

A norma IN1413 Segurança de Acesso em Sistemas Críticos especifica os requisitos para acessos a sistemas.

Todos os usuários deverão observar os procedimentos de ``login'' aprovados para acesso aos equipamentos e sempre que se ausentarem de sua estação de trabalho deverão bloqueá-la ou executar os procedimentos de ``logoff''.

A execução de comandos no sistema operacional dos servidores será restrita a pessoas devidamente autorizadas a executar funções de administração e controle destes sistemas. As regras sobre operacionalidades do controle de acesso estão definidas na norma IN3382 Segurança Lógica.

\subsubsubsection{Controle de acesso a sistemas aplicativos}  

Os acessos realizados deverão ser registrados e para os sistemas com transações críticas devem ser monitorados para identificar potencial mau uso de sistemas ou informações.

As normas IN-SG014-2013 - Segurança de Acesso em Sistemas Críticos e IN2238 Segurança Lógica especificam os requisitos para acessos à rede e sistemas.

\subsubsection{Controle de Acesso Físico}

\subsubsubsection{Ambiente de informática}  

Os equipamentos de informática devem estar em ambiente protegido para a sua melhor operacionalidade e não permitir acesso a pessoas não autorizadas, conforme trata a norma ``IN0602 Acesso Físico às Dependências da Financeira XPTO''.

A sala dos servidores (Datacenter) é o coração tecnológico da empresa. Este local deve ser restrito apenas à equipe de manutenção previamente autorizada. O acesso a esse ambiente deve ser controlado com técnicas de identificação e autenticação fortes com dispositivos específicos tais como monitoração por circuito fechado de TV, sensoriamento interno e externo ininterrupto contra tentativas de acesso não autorizado, conforme trata a norma ``IN0062 Acesso Físico às dependências da Financeira XPTO''.

\subsection{Acesso de Prestadores de Serviços}

Onde existir a necessidade de negócio para acesso de prestadores de serviços, deverá ser feito um acordo prévio por meio de contrato assinado com os mesmos. O contrato deve estabelecer claramente as regras e condições para as permissões de acesso de seus representantes e a confidencialidade das informações criadas ou acessadas, para que não existam divergências de entendimento entre a Financeira XPTO e seus prestadores de serviço, conforme trata a norma ``IN8062 Acesso físico às dependências da Financeira XPTO''.

Nenhum prestador de serviço poderá entrar em qualquer recinto sem obedecer às seguintes regras:

\begin{itemize}
    \item Uso de crachás de identificação;
    \item Registro do visitante ou prestador de serviço, com os dados pessoais, motivo da visita, empresa para a qual trabalha, funcionário visitado ou responsável pela autorização e data e hora de entrada e saída;
    \item Acompanhamento por um funcionário da própria área, já dentro da instalação.
\end{itemize}

\subsection{Proteção contra Invasão da Rede}

Os equipamentos, sistemas operacionais e aplicativos, sistemas de rede e comunicação estarão adequadamente configurados e protegidos contra acesso físico e acesso não autorizado pela rede, conforme trata a norma ``IN9982 - Segurança Lógica''.

Caberá à Gerência de Infraestrutura e Suporte TI proteger a rede corporativa da Financeira XPTO contra invasão interna e externa de usuários não autorizados, para proteção do acervo de dados, do equipamento e da reputação da Financeira XPTO. A proteção da rede de computadores deverá atentar para tipos de ataques como:

\begin{itemize}
    \item Acesso não autorizado;
    \item Impedimento do uso do equipamento;
    \item Roubo de informações;
    \item Indisponibilidade de serviços.
\end{itemize}

Para proteção contra invasão da rede devem ser tomadas as ações necessárias para minimizar riscos e impactos à segurança da informação conforme trata a norma IN2002 Segurança Lógica.

\subsubsection{Execução de Cópias de Segurança}

As cópias dos arquivos mantidos em ambiente de rede serão executadas pelos operadores de rede (IN8202 Segurança Lógica), devendo ser guardadas em local seguro, geograficamente distante dos originais, sempre que for possível, pelo prazo determinado pela respectiva legislação cível e fiscal, quando for o caso.

Periodicamente deverão ser efetuados testes de restauração das cópias de segurança, implicando na substituição das mídias quando necessário.

\subsubsection{Controle de Bibliotecas de Software}

Os softwares de maneira geral deverão estar protegidos e armazenados organizadamente em bibliotecas específicas para ambiente de desenvolvimento/homologação e de produção.

As atualizações das bibliotecas de programas, principalmente as da produção, deverão ser executadas por ``bibliotecário'' autorizado pela Gerência Infraestrutura e Suporte TI, adotando-se os procedimentos descritos na norma IN1201 Gestão de Mudanças.

Os sistemas aplicativos deverão ser disponibilizados aos usuários somente após testes de validação, pelos solicitantes e usuários finais, devidamente aprovados e aceitados em ambiente de homologação. 

Sempre que existir atualização de versão de software, as versões anteriores deverão ser retidas como medida de contingência, clara e formalmente identificadas, com a data da substituição e período pelo qual o programa esteve em produção.

Os dados de teste deverão ser protegidos e controlados. Testes de sistema e de aceitação requerem volumes substanciais de dados e devem refletir, o mais próximo possível, os dados em produção. Entretanto, a base de dados de produção quando disponibilizada para testes seja descaracterizada antes do uso.

Para reduzir o risco potencial de falhas em programas de computadores, deve-se estabelecer um controle do acesso às bibliotecas de programa-fonte, contemplando tanto o ambiente de teste como o de produção.

A distribuição de programa-fonte, para manutenção por programadores e analistas, somente deverá ser efetuada pelo responsável designado em manter a biblioteca e sob autorização do gestor autorizado pela Superintendência de Tecnologia da Informação.

\subsubsection{Locais Físicos para Operação em Situação de Contingência}

A Financeira XPTO deve manter ambientes fisicamente distantes e redundantes para assegurar a restauração da operação em produção em prazo aceitável pelo negócio, conforme o IN2011 Manual do SGCN.

Os ambientes de desenvolvimento e produção também devem ser segregados conforme a rede corporativa.

\subsubsection{Auditorias}

A Financeira XPTO deve planejar, implementar e manter um programa de auditoria, incluindo a frequência, métodos, responsabilidades, requisitos de planejamento e relatórios.

Os programas de auditoria devem levar em conta a importância dos processos pertinentes e os resultados de auditorias anteriores.

Deve ser observada a independência da equipe de auditoria em relação à cadeia hierárquica e responsabilidade da comunicação de fatos relevantes para a Liderança e Gestão da Financeira XPTO.

\subsubsection{Responsabilidades pela Operacionalização da Política de Segurança}

Cabe a todos os funcionários, terceiros e partes interessadas, a responsabilidade pela correta utilização dos equipamentos e das informações geradas, manipuladas e/ou armazenadas, assim como o cumprimento das normas e procedimentos instituídos para cada assunto.

O cumprimento das políticas e normas de segurança é obrigatório e o seu não cumprimento ou a recusa em fazê-lo pode implicar em sanções contratuais, disciplinares ou trabalhistas cabíveis ao infrator.

\subsection{Acesso Remoto na Financeira XPTO}

O acesso remoto é um recurso que permite acesso a sistemas da Financeira XPTO fora de suas dependências físicas. Este acesso está disponível somente através da Internet.

A política de acesso remoto na Financeira XPTO (IN2310 Acesso Remoto na Financeira XPTO) visa orientar a correta utilização das informações recebidas e enviadas pelos funcionários, contratados e parceiros de negócio da empresa através deste recurso, de modo a evitar riscos de exposição que causem dano de toda espécie ao negócio da Financeira XPTO.

\section{Glossário de Termos Específicos Deste Documento}

\begin{longtable}{|>{\centering\arraybackslash}m{3cm}|>{\raggedright\arraybackslash}m{12cm}|}
\hline
\textbf{TERMO} & \textbf{DESCRIÇÃO} \\
\hline
\endfirsthead

\hline
\textbf{TERMO} & \textbf{DESCRIÇÃO} \\
\hline
\endhead

\hline
\endfoot

Backup & Cópia de Segurança. Armazenagem das informações em mídia destinada à contingência de dados. \\
\hline
Backup Site & CPD Secundário. Centro de Processamento de Dados Secundário - Ambiente “Cópia de Segurança” secundário conectado na forma de “espelhamento remoto de dados” com o CPD principal e tem por objetivo entrar em operação imediatamente em casos de paralisação do CPD Primário. É também conhecido como “Hot Backup”. \\
\hline
Firewall & 1- Sistema destinado à proteção lógica de uma rede. 2- Software e/ou hardware que fornece segurança ao isolar a rede de computador de uma empresa do resto da Internet. Um firewall também impede que pessoas de fora conectem a rede; quem estiver interessado em manter o sigilo da informação deve ter um.  
*Fonte: Glossário de TI em \url{http://www.eb.ufmg.br/bax/TecnoInfo/glossario/TI.htm} \\
\hline
Password & Senha. Chave de acesso para um sistema. \\
\hline
No-Break & Apenas no Brasil esse produto é conhecido como No-Break, em todo o mundo ele é conhecido como UPS (Uninterruptible Power Supply). E é exatamente isso que ele é, uma fonte de alimentação ininterrupta de energia elétrica. Seu principal objetivo é proteger sua carga (PCs, Servidores, Periféricos ou outros equipamentos eletrônicos) contra defeitos ou falhas no fornecimento da energia elétrica. \\
\hline
Worms & Um worm, tal como um vírus, foi desenvolvido para copiar a si próprio de computador em computador, mas um worm faz isso automaticamente a apoderar-se de funcionalidades no computador capazes de transportar arquivos ou informações. Um worm que esteja no seu sistema consegue propagar-se sem intervenção do usuário. Um dos grandes perigos dos worms é a sua capacidade de se replicar em grandes volumes, podendo consumir memória, fazendo com que um computador pare de funcionar corretamente. \\
\hline
SPAM & SPAM é o termo usado para se referir aos e-mails não solicitados, que geralmente são enviados para um grande número de pessoas. Quando o conteúdo é exclusivamente comercial, este tipo de mensagem também é referenciado como UCE (do inglês Unsolicited Commercial Email).  
Fonte: \url{http://www.nbso.nic.br/docs/cartilha/cartilha-06-spam.html} \\
\hline
Phishing Scam & Phishing é uma técnica utilizada para obter informação pessoal com propósitos de furto ou fraude através de uso fraudulento de mensagens de emails que aparentam ser de fontes legítimas. Estas mensagens aparentemente autênticas são criadas para enganar os destinatários para divulgarem informações pessoais, tais como: senhas de banco, nºs de cartão de crédito e etc.  
Fonte: \url{http://www.computerworld.com/securitytopics/security/story/0,10801,89096,00.html} e  
\url{http://www.microsoft.com/athome/security/email/phishing.mspx} \\
\hline
Pen drive & Pen drive é um dispositivo com memória flash e conector USB que funciona como unidade de armazenamento removível. Basta plugá-la na porta USB do computador para que seja reconhecida como uma nova unidade de armazenamento pronta para ser utilizada.  
A memória pen drive também é conhecida por outros nomes: memory key, chaveiro USB, flash drive, flash memory, mini HD, entre outros. No Brasil é popularmente conhecida pelo nome de pen drive.  
A grande vantagem desse dispositivo é ser compacto (tamanho aproximado de um chaveiro) com a possibilidade de ter uma grande capacidade de armazenamento, você pode transportá-la para qualquer lugar e plugá-la em qualquer computador com uma porta USB. Podemos dizer que a pen drive tem a função de um mini HD removível.  
Fonte: \url{http://www.pendrivenet.com.br/} \\
\hline
Dispositivo móvel & A definição de dispositivo móvel é qualquer equipamento ou periférico eletrônico, operando com bateria ou não, que possa ser transportado com facilidade e tenha a capacidade de processar, armazenar ou transmitir dados e esteja acessível em qualquer lugar.  
Exemplos de dispositivos móveis: laptops, notebooks, personal digital assistants (PDAs), Blackberry, dispositivos USB (Universal Serial Bus), dispositivos portáteis de rede sem fio (handheld wireless devices), dispositivos USB (Universal Serial Bus), Compact Discs (CDs), Digital Versatile Discs (DVDs), flash drives, MP3/MP4 players, câmeras digitais, telefones com câmera. \\



\end{longtable}

\section{Controle da Documentação}

\subsection{Históricos de Atualização}
\label{sec:historico_de_atualizacao}

\begin{longtable}{|c|c|c|p{6cm}|c|c|}
\hline
\textbf{Versão} & \textbf{Rev} & \textbf{Emissão} & \textbf{Motivo/Descrição} & \textbf{Responsável} & \textbf{Aprovação} \\
\hline
\endfirsthead

\hline
\textbf{Versão} & \textbf{Rev} & \textbf{Emissão} & \textbf{Motivo/Descrição} & \textbf{Responsável} & \textbf{Aprovação} \\
\hline
\endhead

\hline
\endfoot

01 & 00 & 21.03.2002 & Elaboração Inicial & QSCI & 21.03.2002 \\
\hline
02 & 00 & 04.01.2004 & Primeira Revisão Periódica & QSCI & 23.01.2004 \\
\hline
03 & 00 & 17.01.2005 & Segunda Revisão Periódica & QSCI & 17.01.2005 \\
\hline

\end{longtable}

\subsection{Ciclo de Revisão}

Este documento será revisto e atualizado quando:

\begin{itemize}
    \item Houver solicitação de atendimento, correção ou adição de informações;
    \item Existir a necessidade de atender requisitos legais, boas práticas ou recomendações de auditoria;
    \item Existir mudança na organização que tenha impacto relevante na atividade abordada neste documento;
    \item No vencimento, conforme item \hyperref[sec:historico_de_atualizacao]{\ref{sec:historico_de_atualizacao} - \nameref{sec:historico_de_atualizacao}} deste documento.
\end{itemize}

\subsection{Guarda e Retenção}

As versões deste documento deverão ser armazenadas por cinco anos, após o vencimento de seu prazo de validade.

\subsection{Classificação da Segurança}

Este documento é de caráter corporativo. É permitida a reprodução, cópia ou movimentação para fora das dependências da empresa somente com autorização formal do responsável pelo documento ou pessoa por ele designada.

\vspace{2cm}

\noindent
\textbf{Organização XPTO} \\
Brasília, 15 de março de 202x.

\end{document}